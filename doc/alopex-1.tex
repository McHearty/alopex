\documentclass{latex2man}

\begin{Name}{1}{alopex}{Jesse McClure}{window manager}{Alopex - A Tabbed, Tiling Window Manager with Fur}

\Prog{Alopex} - A Tabbed, Tiling Window Manager with Fur

\end{Name}

\section{Synopsis}
xinit alopex \oArg{theme}

\section{Description}

\Prog{Alopex} is a dynamic tiling window manager with a container
concept inspired by i3 but implemented quite differently.  Client
windows are not statically assigned to a given container, instead,
containers each have a configurable maximum number of clients, and
clients fill each container in sequence.  Within containers, multiple
clients are displayed as tabs.

There are three container layouts: rstack, bstack, and
monocle.  Stack modes divide the screen into two sections, the master
region and the stack.  Monocle uses the full area for one container.
All layouts stack clients within containers displaying a tab for each
client in the container's title bar.  These layouts are supplimented by
per-client floating and full-screen modes.

\Prog{Alopex} uses a tag concept rather than the more common
workspace/desktop metaphor.  The default configuration, however, is
optimized for a workspace-like work flow.  This can be modified as
desired.

\section{Configuration}

Configuration is implemented via an X resources data base file which is
read on startup.  A well-commented example configuration file is
distributed with \Prog{alopex} and can be found at
\File{/usr/share/alopex/config}.  Further information on customizing and
configuring alopex can be found on the github wiki.


\section{Status Input}

A child process specified in the configuration file can be launched to
provide status input to be displayed along side the tabs in containers'
bars.  This child process should send text information to its standard
output which will then be read in by \Prog{alopex}.

Each line must contain three ampersands which delineate four text
segments.  The first two segments are always displayed in the first
container's bar: the first before the tag indicators, the second
immediately after the tag indicators.  The third segment is displayed at
the right side of the second container's bar (or the right of the first
container if there is only one visible).  The fourth segment is
displayed at the right side of the last visible container.

Plain text is printed as-is into the above-specified status bar region.
Text within curly-braces is interpreted as a special command.  These
commands can specify a foreground color, set a font, display an icon
or paint the foreground color with an icon mask.

\subsection{Status Commands}

\begin{description}
\item[\Opt{R G B A}]
	Set color with Red Green Blue and Alpha values specified as floating
	point numbers from 0 to 1
\item[\OptArg{i}{#}]
	Draw icon number # in full color
\item[\OptArg{I}{#}]
	Use icon number # as mask for the currently selected color
\item[\Opt{f}]
	Select default font
\item[\Opt{F}]
	Select bold font

\section{Author}
Copyright \copyright 2012-2014 Jesse McClure \\
License GPLv3: GNU GPL version 3 \URL{http://gnu.org/licenses/gpl.html} \\
This is free software: you are free to change and redistribute it. \\
There is NO WARRANTY, to the extent permitted by law.

Submit bug reports via github: \\
\URL{http://github/com/TrilbyWhite/alopex.git}

I would like your feedback.  If you enjoy \Prog{Alopex} see the bottom
of the site below for detauls on submitting comments: \\
\URL{http://mccluresk9.com/software.html}

Documentation added and edited collaboratively by Jesse McClure and Sam
Stuewe.  Any concerns or comments regarding documentation may be
submitted to \Email{halosghost@archlinux.info}.

\section{See Also}
The most up to date documentation can be found at
\URL{https://github.com/TrilbyWhite/alopex/wiki}

\LatexManEnd
